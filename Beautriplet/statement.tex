\documentclass{article}
\usepackage{listings}
\usepackage{geometry}
\geometry{ a4paper, total={170mm,257mm}, top=10mm, right=20mm, bottom=20mm, left=20mm, headheight=67pt, includeheadfoot}
\usepackage{amsmath}
\usepackage{graphicx}
\usepackage{hyperref}
\usepackage{multicol}
\usepackage{fancyhdr}
\usepackage{zref-totpages}
\usepackage{hyperref}
\pagestyle{fancy}
\usepackage{enumitem}
\graphicspath{ {./assets/} }
\hypersetup{ colorlinks=true, linkcolor=black, filecolor=magenta, urlcolor=cyan}
\setlength{\parindent}{0pt}
\setlength{\parskip}{1em}
\setlist{nolistsep}
\fancyhf{} 
\fancyhead[C]{\hspace*{\fill}\includegraphics[height=62pt]{header.png}\hspace*{\fill}}
\fancyfoot[L]{\textbf{B - Beautriplet}}
\fancyfoot[R]{\thepage} 

\renewcommand{\headrulewidth}{0.4pt}
\lstset{
    basicstyle=\ttfamily\small,
    columns=fixed,
    extendedchars=true,
    breaklines=true,
    tabsize=2,
    prebreak=\raisebox{0ex}[0ex][0ex]{\ensuremath{\hookleftarrow}},
    frame=none,
    showtabs=false,
    showspaces=false,
    showstringspaces=false,
    prebreak={},
    keywordstyle=\color[rgb]{0.627,0.126,0.941},
    commentstyle=\color[rgb]{0.133,0.545,0.133},
    stringstyle=\color[rgb]{01,0,0},
    captionpos=t,
    escapeinside={(\%}{\%)}
}
\begin{document}
\begin{center}
    \section*{B - Beautriplet} % ganti judul soal
    Batas Waktu : 1s \\  % ganti time limit
    Batas Memori :  256MB \\ % ganti memory limit
\end{center}

\subsection*{Deskripsi}

Lili memiliki dua buah array, yaitu array $A$ dan $B$. Kedua array tersebut adalah permutasi dengan $N$ bilangan. Karena Lili adalah orang yang ceroboh, kedua array tersebut tidak sengaja tercampur. Hasil campuran kedua array tersebut dinamakan $C$ dengan $2N$ buah elemen. 
Lili menyukai \textbf{triplet cantik} dari array $C$ yang dimilikinya. Suatu triplet $(C_i, C_j, C_k)$ dari array $C$ disebut cantik jika dan hanya jika: 
\begin{enumerate}
    \item $1\le i<j<k\le N$
    \item $C_i=C_k$
\end{enumerate}
Suatu triplet cantik $(C_i, C_j, C_k)$ akan memiliki nilai kecantikan $X$, yaitu $$X=C_i+C_j+C_k$$ Lili ingin mengetahui, berapakah jumlah nilai kecantikan dari seluruh triplet cantik \textbf{unik} yang berada pada array $C$?

\subsection*{Format Masukan}

Baris pertama terdiri dari satu bilangan bulat positif $N$ ($1\le N\le 10^5$), yang menyatakan banyak elemen array $A$ dan $B$.

Baris kedua terdiri dari $2N$ buah bilangan bulat positif $C_i$ $(1 \leq C_i \leq 10^5, 1 \leq i \leq N)$, yang menyatakan elemen ke-$i$ dari array $C$, di mana $C$ merupakan gabungan dari dua permutasi.

\subsection*{Format Keluaran}

Keluarkan jumlah nilai kecantikan dari seluruh triplet cantik yang \textbf{unik}.
\\

\begin{multicols}{2}
\subsection*{Contoh Masukan 1}
\begin{lstlisting}
3
1 2 2 3 1 3
\end{lstlisting}
\columnbreak
\subsection*{Contoh Keluaran 1}
\begin{lstlisting}
16
\end{lstlisting}
\vfill
\null
\end{multicols}


\subsection*{Penjelasan}

Seluruh triplet cantik pada array $C$ adalah:
\begin{enumerate}
    \item $(1,2,1)$ dengan nilai kecantikan $X=1+2+1=4$
    \item $(1,3,1)$ dengan nilai kecantikan $X=1+3+1=5$
    \item $(3,1,3)$ dengan nilai kecantikan $X=3+1+3=7$
\end{enumerate}
Maka, jumlah nilai kecantikan dari seluruh triplet cantik unik adalah $X=4+5+7=16$.

%
%\subsection*{Penjelasan}
%Subsection ini bersifat \textbf{opsional}. %Gunakan subsection ini jika problem dirasa %cukup kompleks atau ketika contoh yang %diberikan dirasa kurang jelas.
%

\end{document}