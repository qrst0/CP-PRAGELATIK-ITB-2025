\documentclass{article}
\usepackage{listings}
\usepackage{geometry}
\geometry{ a4paper, total={170mm,257mm}, top=10mm, right=20mm, bottom=20mm, left=20mm, headheight=67pt, includeheadfoot}
\usepackage{amsmath}
\usepackage{graphicx}
\usepackage{hyperref}
\usepackage{multicol}
\usepackage{fancyhdr}
\usepackage{zref-totpages}
\usepackage{hyperref}
\pagestyle{fancy}
\usepackage{enumitem}
\graphicspath{ {./assets/} }
\hypersetup{ colorlinks=true, linkcolor=black, filecolor=magenta, urlcolor=cyan}
\setlength{\parindent}{0pt}
\setlength{\parskip}{1em}
\setlist{nolistsep}
\fancyhf{} 
\fancyhead[C]{\hspace*{\fill}\includegraphics[height=62pt]{header.png}\hspace*{\fill}}
\fancyfoot[L]{\textbf{E - bbaakktteerrii}}
\fancyfoot[R]{\thepage} 

\renewcommand{\headrulewidth}{0.4pt}
\lstset{
    basicstyle=\ttfamily\small,
    columns=fixed,
    extendedchars=true,
    breaklines=true,
    tabsize=2,
    prebreak=\raisebox{0ex}[0ex][0ex]{\ensuremath{\hookleftarrow}},
    frame=none,
    showtabs=false,
    showspaces=false,
    showstringspaces=false,
    prebreak={},
    keywordstyle=\color[rgb]{0.627,0.126,0.941},
    commentstyle=\color[rgb]{0.133,0.545,0.133},
    stringstyle=\color[rgb]{01,0,0},
    captionpos=t,
    escapeinside={(\%}{\%)}
}
\begin{document}
\begin{center}
    \section*{E - bbaakktteerrii} % ganti judul soal
    Batas Waktu : 1s \\  % ganti time limit
    Batas Memori :  256MB \\ % ganti memory limit
\end{center}

\subsection*{Deskripsi}

Di negeri Bojongsantos, kini terdapat \textit{n} jenis bakteri. Bakteri-bakteri ini memiliki perilaku unik tergantung pada musim di negeri yang ia tinggali. Di negeri ini, terdapat dua musim dalam satu tahun yaitu musim lato-lato dan musim jedag-jedug. Pada musim pertama, musim lato-lato setiap jumlah bakteri akan bertambah menjadi \textit{c}$_{i}$ kali lipat jumlahnya di awal tahun. Kemudian pada musim jedag-jedug ia akan bermutasi ke jenis paling muda yang lebih tua dari jenisnya sejumlah \textit{d}$_{i}$/\textit{c}$_{i}$ dari keseluruhan jumlahnya (jumlah setelah bertambah pada musim lato-lato). Pada musim ini, \textit{d}$_{i}$/\textit{c}$_{i}$ bagian bakteri yang paling tua akan gagal bermutasi dan mati. Jumlah keseluruhan dari tiap-tiap bakteri pada masa kini sedang dicari tahu oleh warga Bojongsantos. Namun jumlah yang mereka ketahui hanyalah jumlah awal kemunculan dari tiap jenis bakteri \textit{a}$_{i}$ ini dan umur dari jenis-jenis bakteri tersebut \textit{b}$_{i}$.

Bantulah warga Bojongsantos menentukan jumlah dari tiap-tiap bakteri tersebut!

\subsection*{Format Masukan}
Baris pertama berisikan satu buah integer ($ 1 \leq \text{N} \leq 10 $)
. Kemudian diikuti oleh N baris berikutnya dimana baris ke-i berisikan empat buah integer \textit{a}$_{i}$ \textit{b}$_{i}$ \textit{c}$_{i}$ \textit{d}$_{i}$ ($ 1 \leq a_{i},b_{i} \leq 10^{7}; 1 \leq c_{i} \leq 10^{3}; 0 \leq d_{i} \leq 10^{3}-1; d_{i} < c_{i}, b_{i} > b_{j} \text{ untuk setiap } i < j $)

\subsection*{Format Keluaran}

Keluarkan N baris berisi satu integer. Dimana integer pada baris ke-i menyatakan jumlah bakteri ke-i pada masa kini dimodulo dengan $10^{9}+7$.
\\

\begin{multicols}{2}
\subsection*{Contoh Masukan 1}
\begin{lstlisting}
3
1 5 3 1
5 3 2 1
10 2 1 0
\end{lstlisting}
\columnbreak
\subsection*{Contoh Keluaran 1}
\begin{lstlisting}
67
5
10
\end{lstlisting}
\vfill
\null
\end{multicols}

%
%\subsection*{Penjelasan}
%
%Penjelasan
%

%
%\subsection*{Penjelasan}
%Subsection ini bersifat \textbf{opsional}. %Gunakan subsection ini jika problem dirasa %cukup kompleks atau ketika contoh yang %diberikan dirasa kurang jelas.
%

\end{document}