\documentclass{article}
\usepackage{listings}
\usepackage{geometry}
\geometry{ a4paper, total={170mm,257mm}, top=10mm, right=20mm, bottom=20mm, left=20mm, headheight=67pt, includeheadfoot}
\usepackage{amsmath}
\usepackage{graphicx}
\usepackage{hyperref}
\usepackage{multicol}
\usepackage{fancyhdr}
\usepackage{zref-totpages}
\usepackage{hyperref}
\pagestyle{fancy}
\usepackage{enumitem}
\graphicspath{ {./assets/} }
\hypersetup{ colorlinks=true, linkcolor=black, filecolor=magenta, urlcolor=cyan}
\setlength{\parindent}{0pt}
\setlength{\parskip}{1em}
\setlist{nolistsep}
\fancyhf{} 
\fancyhead[C]{\hspace*{\fill}\includegraphics[height=62pt]{header.png}\hspace*{\fill}}
\fancyfoot[L]{\textbf{D - Darwin dan Dungeon}}
\fancyfoot[R]{\thepage} 

\renewcommand{\headrulewidth}{0.4pt}
\lstset{
    basicstyle=\ttfamily\small,
    columns=fixed,
    extendedchars=true,
    breaklines=true,
    tabsize=2,
    prebreak=\raisebox{0ex}[0ex][0ex]{\ensuremath{\hookleftarrow}},
    frame=none,
    showtabs=false,
    showspaces=false,
    showstringspaces=false,
    prebreak={},
    keywordstyle=\color[rgb]{0.627,0.126,0.941},
    commentstyle=\color[rgb]{0.133,0.545,0.133},
    stringstyle=\color[rgb]{01,0,0},
    captionpos=t,
    escapeinside={(\%}{\%)}
}
\begin{document}
\begin{center}
    \section*{D - Darwin dan Dungeon} % ganti judul soal
    Batas Waktu : 1s \\  % ganti time limit
    Batas Memori :  256MB \\ % ganti memory limit
\end{center}

\subsection*{Deskripsi}

Suatu hari, Darwin sedang berpetualang ke dalam sebuah \textit{dungeon}. Saat hendak masuk ke dalam ruangan harta, seorang penjaga \textit{dungeon} menghalangi Darwin. Penjaga tersebut memberikan sebuah syarat agar Darwin dapat masuk ke dalam ruangan harta karun, yaitu menjawab $Q$ buah kueri yang diberikan dengan benar. 

Untuk setiap kueri, Darwin akan diberikan dua buah bilangan bulat positif $L$ dan $R$. Tugas Darwin adalah menjawab banyak bilangan prima berbeda yang berada pada interval $[L, R]$. Sayangnya, Darwin sudah kelelahan setelah menghadapi banyak monster sehingga ia meminta bantuan Anda untuk menjawabnya.

\subsection*{Format Masukan}
Baris pertama berisi bilangan bulat $Q$ ($1 \leq Q \leq  10^5 $) yang mewakili banyak kueri yang diberikan.

Sebanyak $Q$ baris berikutnya berisi bilangan bulat $L$ dan $R$ ($ 1 \le L \le R \le 10^{6} $).

\subsection*{Format Keluaran}

Keluarkan $Q$ baris, dengan baris ke-$i$ menandakan jawaban dari kueri ke-$i$.

\begin{multicols}{2}
\subsection*{Contoh Masukan 1}
\begin{lstlisting}
2
1 10
11 15
\end{lstlisting}
\columnbreak
\subsection*{Contoh Keluaran 1}
\begin{lstlisting}
4
2
\end{lstlisting}
\vfill
\null
\end{multicols}


\subsection*{Penjelasan}

Pada interval $[1, 10]$, terdapat 4 buah bilangan prima berbeda, yaitu $\{2, 3, 5, 7\}$.

Pada interval $[11, 15]$, terdapat 2 buah bilangan prima berbeda, yaitu $\{11, 13\}$.

\end{document}