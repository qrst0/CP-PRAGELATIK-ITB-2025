\documentclass{article}
\usepackage{listings}
\usepackage{geometry}
\geometry{ a4paper, total={170mm,257mm}, top=10mm, right=20mm, bottom=20mm, left=20mm, headheight=67pt, includeheadfoot}
\usepackage{amsmath}
\usepackage{graphicx}
\usepackage{hyperref}
\usepackage{multicol}
\usepackage{fancyhdr}
\usepackage{zref-totpages}
\usepackage{hyperref}
\pagestyle{fancy}
\usepackage{enumitem}
\graphicspath{ {./assets/} }
\hypersetup{ colorlinks=true, linkcolor=black, filecolor=magenta, urlcolor=cyan}
\setlength{\parindent}{0pt}
\setlength{\parskip}{1em}
\setlist{nolistsep}
\fancyhf{} 
\fancyhead[C]{\hspace*{\fill}\includegraphics[height=62pt]{header.png}\hspace*{\fill}}
\fancyfoot[L]{\textbf{I - Iyan}}
\fancyfoot[R]{\thepage} 

\renewcommand{\headrulewidth}{0.4pt}
\lstset{
    basicstyle=\ttfamily\small,
    columns=fixed,
    extendedchars=true,
    breaklines=true,
    tabsize=2,
    prebreak=\raisebox{0ex}[0ex][0ex]{\ensuremath{\hookleftarrow}},
    frame=none,
    showtabs=false,
    showspaces=false,
    showstringspaces=false,
    prebreak={},
    keywordstyle=\color[rgb]{0.627,0.126,0.941},
    commentstyle=\color[rgb]{0.133,0.545,0.133},
    stringstyle=\color[rgb]{01,0,0},
    captionpos=t,
    escapeinside={(\%}{\%)}
}
\begin{document}
\begin{center}
    \section*{I - Iyan } % ganti judul soal
    Batas Waktu : 1s \\  % ganti time limit
    Batas Memori :  256MB \\ % ganti memory limit
\end{center}

\subsection*{Deskripsi}

Iyan adalah seorang ratu pada suatu kota di sebuah kerajaan. Pada kerajaan ini, kekuatan dari suatu kota diukur dari kecantikan pasukan yang dimiliki oleh ratu. Iyan memiliki $N$ buah pasukan, pasukan ke-$i$ memiliki kecantikan sebesar $A_i$ (Perhatikan bahwa $A_i$ dapat bernilai negatif).

Karena Iyan adalah ratu yang penuh dengan ego, ia ingin menambahkan kecantikan pasukan-pasukannya agar dapat menjadi kota tercantik pada kerajaan. Namun, karena Iyan juga ceroboh, terkadang ia tidak sengaja malah mengurangi kecantikan pasukan-pasukannya.

Iyan akan memberikan $Q$ buah perintah. Terdapat 2 jenis perintah yang akan diberikan Iyan. Yang pertama, Iyan akan memberikan perintah untuk menambahkan kecantikan sebesar $K$ pada seluruh pasukan dengan kecantikan $X$. Yang kedua, karena Iyan juga pelupa, ia ingin mengetahui kecantikan pasukan ke-$i$ saat ini.

Secara formal, terdapat 2 jenis perintah yang akan diberikan Iyan:
\begin{itemize}
    \item $1$ $X$ $K$ : Menambahkan kecantikan sebesar $K$ pada seluruh pasukan dengan kecantikan $X$.
    \item $2$ $i$ : Menampilkan kecantikan pasukan ke-$i$ ($A_i$) saat ini.
\end{itemize}

\subsection*{Format Masukan}

Baris pertama berisi dua bilangan bulat positif $N$ dan $Q$ ($1 \leq N, Q \leq 10^5$) yang menyatakan banyaknya pasukan yang dimiliki Iyan dan banyaknya perintah yang akan diberikan Iyan.

Baris kedua berisi $N$ buah bilangan bulat $A_1, A_2, \dots, A_N$ ($-10^5 \leq A_i \leq 10^5$) yang menyatakan kecantikan pasukan ke-$i$.

$Q$ baris berikutnya berisi salah satu dari dua jenis perintah yang akan diberikan Iyan sebagai berikut:
\begin{itemize}
    \item $1$ $X$ $K$ ($-10^{12} \leq X \leq 10^{12}$, $-10^5 \leq K \leq 10^5$) yang menyatakan perintah untuk menambahkan kecantikan sebesar $K$ pada seluruh pasukan dengan kecantikan $X$.
    \item $2$ $i$ ($1 \leq i \leq N$) yang menyatakan perintah untuk menampilkan kecantikan pasukan ke-$i$ saat ini.
\end{itemize}

\subsection*{Format Keluaran}

Untuk setiap perintah jenis kedua, keluarkan kecantikan pasukan ke-$i$ saat ini.
\\
\pagebreak
\begin{multicols}{2}
\subsection*{Contoh Masukan 1}
\begin{lstlisting}
5 5  
2 3 2 4 3  
1 2 2  
2 1  
1 4 5  
2 3  
2 2  
\end{lstlisting}
\columnbreak
\subsection*{Contoh Keluaran 1}
\begin{lstlisting}
4
9
3
\end{lstlisting}
\vfill
\null
\end{multicols}


\subsection*{Penjelasan}

Kecantikan awal pasukan adalah $[2, 3, 2, 4, 3]$. Kemudian, Iyan memberikan perintah secara berurutan:
\begin{enumerate}
    \item Ubah seluruh pasukan dengan kecantikan $2$ menjadi $2+2=4$. Maka, kecantikan pasukan menjadi $[4, 3, 4, 4, 3]$.
    \item Tampilkan kecantikan pasukan ke-$1$. Kecantikan pasukan ke-$1$ adalah $4$.
    \item Ubah seluruh pasukan dengan kecantikan $4$ menjadi $4+5=9$. Maka, kecantikan pasukan menjadi $[9, 3, 9, 9, 3]$.
    \item Tampilkan kecantikan pasukan ke-$3$. Kecantikan pasukan ke-$3$ adalah $9$.
    \item Tampilkan kecantikan pasukan ke-$2$. Kecantikan pasukan ke-$2$ adalah $3$.
\end{enumerate}

%
%\subsection*{Penjelasan}
%Subsection ini bersifat \textbf{opsional}. %Gunakan subsection ini jika problem dirasa %cukup kompleks atau ketika contoh yang %diberikan dirasa kurang jelas.
%

\end{document}