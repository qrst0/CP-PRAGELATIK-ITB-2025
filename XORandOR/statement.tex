\documentclass{article}
\usepackage{listings}
\usepackage{geometry}
\geometry{ a4paper, total={170mm,257mm}, top=10mm, right=20mm, bottom=20mm, left=20mm, headheight=67pt, includeheadfoot}
\usepackage{amsmath}
\usepackage{graphicx}
\usepackage{hyperref}
\usepackage{multicol}
\usepackage{fancyhdr}
\usepackage{zref-totpages}
\usepackage{hyperref}
\pagestyle{fancy}
\usepackage{enumitem}
\graphicspath{ {./assets/} }
\hypersetup{ colorlinks=true, linkcolor=black, filecolor=magenta, urlcolor=cyan}
\setlength{\parindent}{0pt}
\setlength{\parskip}{1em}
\setlist{nolistsep}
\fancyhf{} 
\fancyhead[C]{\hspace*{\fill}\includegraphics[height=62pt]{header.png}\hspace*{\fill}}
\fancyfoot[L]{\textbf{C - XOR and OR}}
\fancyfoot[R]{\thepage} 

\renewcommand{\headrulewidth}{0.4pt}
\lstset{
    basicstyle=\ttfamily\small,
    columns=fixed,
    extendedchars=true,
    breaklines=true,
    tabsize=2,
    prebreak=\raisebox{0ex}[0ex][0ex]{\ensuremath{\hookleftarrow}},
    frame=none,
    showtabs=false,
    showspaces=false,
    showstringspaces=false,
    prebreak={},
    keywordstyle=\color[rgb]{0.627,0.126,0.941},
    commentstyle=\color[rgb]{0.133,0.545,0.133},
    stringstyle=\color[rgb]{01,0,0},
    captionpos=t,
    escapeinside={(\%}{\%)}
}
\begin{document}
\begin{center}
    \section*{C - XOR and OR} % ganti judul soal
    Batas Waktu : 3s \\  % ganti time limit
    Batas Memori :  256MB \\ % ganti memory limit
\end{center}

\subsection*{Deskripsi}

Anda diberikan dua buah array $A$ dan $B$, keduanya dengan panjang $N$. Untuk sebuah set $S$ sehingga $S \subseteq \{1, \dots, N\}$, definisikan $P(S)$ sebagai set yang terbentuk dari seluruh $A_i$ sehingga $i \in S$. Definisikan juga $Q(S)$ sebagai set yang terbentuk dari seluruh $B_i$ sehingga $i \in S$. 

Misal $T$ adalah sebuah set bilangan bulat positif. Definisikan $R(T)$ sebagai set bilangan positif yang dapat dibentuk dari bitwise XOR elemen himpunan $T$. Contoh, untuk $T = \{2, 3, 5 \}$, $R(T)= \{1,2,3,4,5,6,7\}$. 

Misal $T$ adalah sebuah set bilangan positif. Definisikan $U(T)$ sebagai nilai bitwise OR dari seluruh elemen $T$. Contoh, untuk $T = \{2, 6, 7 \}$, $U(T)= 3$. 

Anda akan diberikan $Q$ kueri. Untuk setiap kueri, Anda akan diberikan sebuah bilangan bulat positif $X$. Untuk setiap kueri, keluarkan nilai dari:
\[
  \min_{S \subseteq \{1, \dots, N \} \wedge X \in R(P(S))} U(Q(S))
\]

Jika tidak ada $S$ yang memenuhi, keluarkan -1.

\subsection*{Format Masukan}
Baris pertama berisi bilangan bulat $N$ ($1 \leq N \leq  10^5 $).

Baris kedua berisi $N$ bilangan bulat yang menandakan array $A$ ($1 \leq A_i < 2^{14}$),

Baris ketiga berisi $N$ bilangan bulat yang menandakan array $B$ ($1 \leq B_i < 2^{15}$).

Baris keempat berisi bilangan bulat $Q$ ($1 \leq Q \leq 10^4$).

Baris kelima berisi $Q$ bilangan bulat yang menandakan kueri $X$ ($1 \leq X_i < 2^{14}$). 

\subsection*{Format Keluaran}

Keluarkan $Q$ baris, dengan baris ke-$i$ menandakan jawaban dari kueri ke-$i$.
\\

\begin{multicols}{2}
\subsection*{Contoh Masukan 1}
\begin{lstlisting}
3
1 2 3
6 5 4
3
2 3 1
\end{lstlisting}
\columnbreak
\subsection*{Contoh Keluaran 1}
\begin{lstlisting}
5
4
5
\end{lstlisting}
\vfill
\null
\end{multicols}


\subsection*{Penjelasan}

Untuk $X = 2$, set $S$ yang memberikan nilai OR terkecil adalah $S = \{2\}$. Begitu juga ketika $X = 3$, set $S$ yang memberikan nilai OR terkecil adalah $S = \{3\}$. Untuk $X = 1$, set paling optimal adalah $S = \{ 2, 3 \}$.

\end{document}