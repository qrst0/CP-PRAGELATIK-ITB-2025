\documentclass{article}
\usepackage{listings}
\usepackage{geometry}
\geometry{ a4paper, total={170mm,257mm}, top=10mm, right=20mm, bottom=20mm, left=20mm, headheight=67pt, includeheadfoot}
\usepackage{amsmath}
\usepackage{graphicx}
\usepackage{hyperref}
\usepackage{multicol}
\usepackage{fancyhdr}
\usepackage{zref-totpages}
\usepackage{hyperref}
\pagestyle{fancy}
\usepackage{enumitem}
\graphicspath{ {./assets/} }
\hypersetup{ colorlinks=true, linkcolor=black, filecolor=magenta, urlcolor=cyan}
\setlength{\parindent}{0pt}
\setlength{\parskip}{1em}
\setlist{nolistsep}
\fancyhf{} 
\fancyhead[C]{\hspace*{\fill}\includegraphics[height=62pt]{header.png}\hspace*{\fill}}
\fancyfoot[L]{\textbf{A - Artefak Maple}}
\fancyfoot[R]{\thepage} 

\renewcommand{\headrulewidth}{0.4pt}
\lstset{
    basicstyle=\ttfamily\small,
    columns=fixed,
    extendedchars=true,
    breaklines=true,
    tabsize=2,
    prebreak=\raisebox{0ex}[0ex][0ex]{\ensuremath{\hookleftarrow}},
    frame=none,
    showtabs=false,
    showspaces=false,
    showstringspaces=false,
    prebreak={},
    keywordstyle=\color[rgb]{0.627,0.126,0.941},
    commentstyle=\color[rgb]{0.133,0.545,0.133},
    stringstyle=\color[rgb]{01,0,0},
    captionpos=t,
    escapeinside={(\%}{\%)}
}
\begin{document}
\begin{center}
    \section*{A - Artefak Maple} % ganti judul soal
    Batas Waktu : 1s \\  % ganti time limit
    Batas Memori :  256MB \\ % ganti memory limit
\end{center}

\subsection*{Deskripsi}

Di dunia kuno Maple, terdapat sebuah artefak legendaris bernama \textit{Harmony Crystal}. Artefak ini diyakini memiliki kekuatan untuk menyeimbangkan dunia dengan menyerap energi dari suatu deretan angka.

Suatu hari, seorang Penjaga Angka bernama Deculein menemukan sebuah gulungan kuno yang berisi sebuah deretan angka mistis. Deretan angka tersebut berisi $N$ buah bilangan bulat positif. Deculein harus menentukan total energi harmonis yang dapat diekstraksi dari angka-angka tersebut. Energi ini dihitung dengan mengambil semua subsekuens tak kosong dari larik, kemudian menjumlahkan faktor persekutuan terbesar (GCD) dari setiap subsekuens tersebut.

Sayangnya, kekuatan \textit{Harmony Crystal} tidak bisa menangani jumlah yang terlalu besar. Karena itu, Deculein harus mengeluarkan hasilnya dalam bentuk modulo $10^9 + 7$. Dapatkah Anda membantu Deculein menyelesaikan tugas ini?

\subsection*{Format Masukan}
Baris pertama berisi bilangan bulat $N$ ($1 \leq N \leq  10^5 $), yaitu jumlah angka pada deretan mistis. 

Baris kedua berisi $N$ buah bilangan bulat $a_{1}, a_{2}, \dots, a_{N} (1 \leq a_{i} \leq 10^6)$, yaitu angka-angka dalam deretan mistis tersebut.

\subsection*{Format Keluaran}

Keluarkan sebuah bilangan bulat, yaitu total energi harmonis yang dihitung dengan menjumlahkan semua GCD dari setiap subsekuens, dengan hasil modulo $10^9 + 7$.

\begin{multicols}{2}
\subsection*{Contoh Masukan 1}
\begin{lstlisting}
3
3 6 9
\end{lstlisting}
\columnbreak
\subsection*{Contoh Keluaran 1}
\begin{lstlisting}
30
\end{lstlisting}
\vfill
\null
\end{multicols}


\subsection*{Penjelasan}

Pada deretan angka $[3, 6, 9]$, semua subsekuen tak kosong yang mungkin adalah:

\begin{itemize}
    \item $[3] \rightarrow GCD = 3$
    \item $[6] \rightarrow GCD = 6$
    \item $[9] \rightarrow GCD = 9$
    \item $[3, 6] \rightarrow GCD = 3$
    \item $[3, 9] \rightarrow GCD = 3$
    \item $[6, 9] \rightarrow GCD = 3$
    \item $[3, 6, 9] \rightarrow GCD = 3$
\end{itemize}

Jadi, total GCD dari setiap subsekuens tak kosong adalah $3 + 6 + 9 + 3 + 3 + 3 + 3 = \textbf{30}$.
\end{document}