\documentclass{article}
\usepackage{listings}
\usepackage{geometry}
\geometry{ a4paper, total={170mm,257mm}, top=10mm, right=20mm, bottom=20mm, left=20mm, headheight=67pt, includeheadfoot}
\usepackage{amsmath}
\usepackage{graphicx}
\usepackage{hyperref}
\usepackage{multicol}
\usepackage{fancyhdr}
\usepackage{zref-totpages}
\usepackage{hyperref}
\pagestyle{fancy}
\usepackage{enumitem}
\graphicspath{ {./assets/} }
\hypersetup{ colorlinks=true, linkcolor=black, filecolor=magenta, urlcolor=cyan}
\setlength{\parindent}{0pt}
\setlength{\parskip}{1em}
\setlist{nolistsep}
\fancyhf{} 
\fancyhead[C]{\hspace*{\fill}\includegraphics[height=62pt]{header.png}\hspace*{\fill}}
\fancyfoot[L]{\textbf{F - Max Flow?}}
\fancyfoot[R]{\thepage} 

\renewcommand{\headrulewidth}{0.4pt}
\lstset{
    basicstyle=\ttfamily\small,
    columns=fixed,
    extendedchars=true,
    breaklines=true,
    tabsize=2,
    prebreak=\raisebox{0ex}[0ex][0ex]{\ensuremath{\hookleftarrow}},
    frame=none,
    showtabs=false,
    showspaces=false,
    showstringspaces=false,
    prebreak={},
    keywordstyle=\color[rgb]{0.627,0.126,0.941},
    commentstyle=\color[rgb]{0.133,0.545,0.133},
    stringstyle=\color[rgb]{01,0,0},
    captionpos=t,
    escapeinside={(\%}{\%)}
}
\begin{document}
\begin{center}
    \section*{F - Max Flow?} % ganti judul soal
    Batas Waktu : 2s \\  % ganti time limit
    Batas Memori :  256MB \\ % ganti memory limit
\end{center}

\subsection*{Deskripsi}

Pada acaca ramadhan, diadakan sebuah festival kurma, dimana pengunjung dapat mendapatkan kurma secara gratis. Untuk memeriahkan festival, terdapat mekanisme unik untuk mengambil kurma.

Terdapat $N$ buah toples kurma disusun secara menyamping, dimana tiap toples tersebut memiliki kapasitas $10^{100}$ kurma, akan tetapi toples ke-$i$ hanya diisi $C_i$ buah kurma. Ketika anda RSVP untuk hadir pada acara festival ini, anda akan mendapatkan $D$ buah tiket. Pada tiket ke-$j$ tersebut, bertuliskan $l_i$, $r_i$, $v_i$. Yang berarti, anda hanya boleh mengakses toples bernomor $l_i$, $l_i+1$, $l_i+2$, ..., $r_i$. Selain itu, banyaknya kurma yang bisa anda tukarkan dengan tiket tersebut tidak boleh melebihi $v_i$. Anda menyadari bahwa tidak ada larangan untuk memindahkan kurma dari toples satu ke toples lainnya, selama toples-toples tersebut adalah toples yang bisa anda akses. 

Dengan menggunakan cara yang optimal, tentukan berapa banyak total kurma yang bisa anda dapatkan.
\subsection*{Format Masukan}

Baris pertama berisi 2 buah bilangan $N, D$ ($1 \leq N, D \leq 10^5$), yang menyatakan banyaknya toples kurma serta banyaknya tiket yang anda dapatkan \textit{mana stone}. 

Baris berikutnya adalah $C_1, C_2, C_3, \dots, C_N$, yaitu banyaknya kurma pada masing masing toples.

D baris berikutnya berisi 3 buah bilangan $l_i, r_i, v_i$ ($1 \leq l_i \leq r_i \leq N, 1 \leq v_i \leq 10^9$) yaitu kurma-kurma yang bisa anda akses serta banyak kurma yang bisa anda tukarkan dengan tiket ke-$i$.

\subsection*{Format Keluaran}
Keluarkan sebuah bilangan yang menyatakan total kurma maksimal yang bisa didapatkan.


\begin{multicols}{2}
\subsection*{Contoh Masukan 1}
\begin{lstlisting}
4 4
5 4 3 2
1 2 4
1 1 3
2 4 1
3 3 4
\end{lstlisting}
\columnbreak
\subsection*{Contoh Keluaran 1}
\begin{lstlisting}
12
\end{lstlisting}
\vfill
\null
\end{multicols}

\subsection*{Penjelasan}
Pada penukaran tiket 1, anda dapat mengambil 2 buah kurma dari toples 1, dan 2 buah kurma dari toples 2. Sehingga kurma pada toples setelah pengambilan menjadi $[3, 2, 3, 2]$

Pada penukaran tiket 2, anda dapat mengambil 3 buah kurma dari toples 1. Sehingga kurma pada toples setelah pengambilan menjadi $0, 2, 3, 2$

Pada penukaran tiket 3, anda dapat mengambil 1 buah kurma pada toples 3, kemudian memindahkan semua kurma pada toples ke-2 dan ke-4 menuju ke toples 3. Sehingga kurma pada toples setelah pengambilan menjadi $[0, 0, 6, 0]$

Pada penukaran tiket 4, anda dapat mengambil 4 buah kurma pada toples ke-3. Sehingga kurma pada toples setelah pengambilan menjadi $[0, 0, 2, 0]$

Sehingga total kurma yang anda dapat adalah 4 + 3 + 1 + 4 = 12

\end{document}