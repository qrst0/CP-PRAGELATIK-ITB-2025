\documentclass{article}
\usepackage{listings}
\usepackage{geometry}
\geometry{ a4paper, total={170mm,257mm}, top=10mm, right=20mm, bottom=20mm, left=20mm, headheight=67pt, includeheadfoot}
\usepackage{amsmath}
\usepackage{graphicx}
\usepackage{hyperref}
\usepackage{multicol}
\usepackage{fancyhdr}
\usepackage{zref-totpages}
\usepackage{hyperref}
\pagestyle{fancy}
\usepackage{enumitem}
\graphicspath{ {./assets/} }
\hypersetup{ colorlinks=true, linkcolor=black, filecolor=magenta, urlcolor=cyan}
\setlength{\parindent}{0pt}
\setlength{\parskip}{1em}
\setlist{nolistsep}
\fancyhf{} 
\fancyhead[C]{\hspace*{\fill}\includegraphics[height=62pt]{header.png}\hspace*{\fill}}
\fancyfoot[L]{\textbf{H - Hitam}}
\fancyfoot[R]{\thepage} 

\renewcommand{\headrulewidth}{0.4pt}
\lstset{
    basicstyle=\ttfamily\small,
    columns=fixed,
    extendedchars=true,
    breaklines=true,
    tabsize=2,
    prebreak=\raisebox{0ex}[0ex][0ex]{\ensuremath{\hookleftarrow}},
    frame=none,
    showtabs=false,
    showspaces=false,
    showstringspaces=false,
    prebreak={},
    keywordstyle=\color[rgb]{0.627,0.126,0.941},
    commentstyle=\color[rgb]{0.133,0.545,0.133},
    stringstyle=\color[rgb]{01,0,0},
    captionpos=t,
    escapeinside={(\%}{\%)}
}
\begin{document}
\begin{center}
    \section*{H - \textbf{Hitam}} % ganti judul soal
    Batas Waktu : 1s \\  % ganti time limit
    Batas Memori :  256MB \\ % ganti memory limit
\end{center}

\subsection*{Deskripsi}

Ciran dan Miris sedang bermain di suatu festival. Di dekat arena permainan, terdapat $N$ buah kotak. Di dalam kotak ke-$i$, terdapat $A_i$ bola putih dan $B_i$ bola \textbf{hitam}. Ketika Miris selesai bermain, Miris akan mengambil satu bola dari kotak ke-$N$. Jika Miris mendapatkan bola \textbf{hitam}, Miris akan mendapatkan hadiah. 

Ketika Miris sedang bermain, Ciran menyelinap ke tempat kotak-kotak tersebut berada, dengan keinginan untuk memperbesar peluang Miris mendapatkan bola \textbf{hitam} nantinya. Untuk setiap kotak (kecuali kotak terakhir), Ciran akan mengambil tepat 1 bola, lalu memindahkannya ke kotak di kanannya. Secara formal, Ciran akan mengambil tepat 1 bola pada kotak ke-$i$, lalu memindahkannya ke kotak ke-$i+1$. 

Namun, Ciran hanya mengetahui jumlah bola putih dan jumlah bola \textbf{hitam} pada suatu kotak. Ketika mengambil bola, Ia tidak dapat melihat warna dari bola tersebut karena buru-buru memindahkannya, agar tidak ketahuan. Setelah semua pemindahan dilakukan, Miris akan mengambil 1 bola dari kotak terakhir (kotak ke-$N$).

Jika Ciran bebas memilih urutan pemindahan bola, misalnya memindahkan bola dari kotak ke‑$3$ ke kotak ke‑$4$ terlebih dahulu, lalu dari kotak ke‑$2$ ke kotak ke‑$3$, tentukan peluang harapan maksimum bahwa bola yang diambil dari kotak ke-$N$ adalah bola \textbf{hitam}!

\subsection*{Format Masukan}

Baris pertama terdiri atas sebuah bilangan bulat positif yaitu $N$, ($1 \leq N \leq 10^6$) yang menyatakan jumlah kotak.

Baris berikutnya terdiri atas $N$ buah angka $A_{i}$ ($0 \leq A_{i} \leq 10^9$) yang menandakan jumlah bola putih pada kotak ke-$i$.

Baris berikutnya terdiri atas $N$ buah angka $B_{i}$ ($0 \leq B_{i} \leq 10^9$) yang menandakan jumlah bola \textbf{hitam} pada kotak ke-$i$.

\subsection*{Format Keluaran}

Peluang harapan maksimum terambilnya bola \textbf{hitam} dari kotak ke-$N$. Jawaban dianggap benar jika memiliki error absolut atau relatif kurang dari $10^6$.
\\

\begin{multicols}{2}
\subsection*{Contoh Masukan 1}
\begin{lstlisting}
2
2 3
5 5
\end{lstlisting}
\columnbreak
\subsection*{Contoh Keluaran 1}
\begin{lstlisting}
0.634921
\end{lstlisting}
\vfill
\null
\end{multicols}

\end{document}